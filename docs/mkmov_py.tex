\documentclass[12pt,dvips]{article}
\usepackage{graphicx}
\usepackage{amsmath}
\usepackage{amssymb}

\setlength{\topmargin}{0.0in}
\setlength{\headheight}{0.0in}
\setlength{\headsep}{0.0in}
\setlength{\oddsidemargin}{0.0in}
\setlength{\textheight}{9.0in}
\setlength{\textwidth}{6.5in}

\newcommand{\cf}{cf.\ }
\newcommand{\eg}{e.g., }
\newcommand{\etal}{et al.}
\newcommand{\ie}{i.e., }
\newcommand{\viz}{viz.\ }
\newcommand{\vs}{vs.\ }

\newcommand{\sect}[1]{Section \ref{s:#1}}
\newcommand{\fig}[1]{Fig.\ \ref{f:#1}}
\newcommand{\eqn}[1]{Eq.\ (\ref{e:#1})}

\newcommand{\code}[1]{\texttt{#1}}

\newcommand{\bvec}[1]{\mathbf{#1}} % vector
\newcommand{\bhat}[1]{\bvec{\hat{#1}}} % unit vector
\newcommand{\bmat}[1]{\bvec{#1}} % matrix
\newcommand{\bsym}[1]{\mbox{\boldmath $#1$}} % symbol (e.g., lowercase Greek)
\newcommand{\vdot}{\bsym{\cdot}} % dot product
\newcommand{\cross}{\bsym{\times}} % cross product
\newcommand{\trans}[1]{#1^\mathrm{T}} % transpose
\newcommand{\inver}[1]{#1^{-1}} % inverse

\newcommand{\hide}[1]{} % to omit text sections

\newcommand{\pkd}{\texttt{pkdgrav}}


\begin{document}

\begin{flushleft}
  \huge{mkmov.py Documentation}\\
  \bigskip\bigskip
  \Large{Last Update: 7/21/18}\\
  \bigskip\bigskip
  \large{Started By: Derek C. Richardson}\\
  \bigskip
  \large{Contact Info:}\\
  Department of Astronomy\\
  University of Maryland\\
  College Park MD 20742\\
  Tel: 301-405-8786\\
  E-mail: \texttt{dcr@astro.umd.edu}
\end{flushleft}

\tableofcontents

\section{OVERVIEW}

This utility automates the process of generating MPEG movies from
\pkd\ output.  The steps are to run \code{ssdraw} on \pkd\ output
files, convert the files using either \code{povray} or \code{rastoppm}
to a format readable by \code{ffmpeg}, then generate the movie.  Each
of these programs must be in the user's search path for this script to
work.  In addition, at a minimum the user must provide
\code{ssdraw.par} to go along with the \pkd\ output files.  If
available, \code{ss.par} is used to determine the \pkd\ output
basename (otherwise ``\code{ss}'' is assumed)---the output files are
assumed to be of the form \code{basename.[0-9]*[0-9]} (\ie the
basename followed by at least 2 digits, representing the timestep) and
optionally with ``\code{.r}'' appended in the case of reduced output.
If invoking POV-Ray (determined by the particle shape in
\code{ssdraw.par}), the user must also provide \code{povray.inc} in
the run directory.

\section{USAGE}

The script can be run without arguments (that's the whole idea!), but
several options are provided for added functionality, as follows:

\begin{description}
\item{\code{--help}} ~ If present, shows a usage message and exits.
\item{\code{--force}} ~ If present, overwrites any existing images.
  Cannot be used with \code{--keep}.  Useful when redoing a movie from
  scratch and old images already exist in the run directory.
\item{\code{--keep}} ~ If present, retains existing images without
  overwriting them.  Cannot be used with \code{--force}.  Useful for
  generating images from newly added output without having to
  regenerate images.  A new movie using the both old and new images
  will be made.
\item{\code{--batch}} ~ If present, runs \code{ssdraw} on all
  \pkd\ output files at once (depending on the \code{--force|--keep}
  flags).  Needed when using \code{ssdraw} functionality that is
  frame-number dependent, such as an inertial camera.  Otherwise
  \code{ssdraw} is run on one file at a time (followed by image
  conversion and cleanup on each).
\item{\code{--color24}} ~ Indicates that the \pkd\ output files use
  24-bit color instead of a color index.  Only valid for POV-Ray
  output.  Useful in conjunction with the \code{sscolor} utility.
\item{\code{--orient}} ~ Indicates that \code{ssdraw} should take
  particle orientation into account when drawing.  Requires \pkd\ to
  be compiled and run with orientation support (making ``\code{.ori}''
  files in addition to regular outputs).  Only valid for POV-Ray
  output.
\item{\code{--quiet}} ~ If present, runs the script in ``quiet'' mode,
  with no output to the terminal unless an error is encountered.
\item{\code{--nth NTH}} ~ Only draws every \code{NTH} frame
  (default 1, \ie draw every frame).  Useful for quickly generating a
  movie.  Even more useful in conjunction with \code{--keep}.
\item{\code{--rate RATE}} ~ Set the movie frame rate to \code{RATE}
  frames per second (default 25, which is the \code{ffmpeg} default).
  Here \code{RATE} is a string expression, so ``1/5'' is interpreted
  as 5 seconds per frame, and, \eg ``30000/1001'' is the proper frame
  rate for NTSC video.  Useful for slowing down or speeding up a movie
  smoothly (use a low frame rate to slow it down, a high frame rate to
  speed it up), or to set the frame rate to ``real time,'' knowing the
  simulation timestep.
\end{description}

Note the above options can be abbreviated to single letters, e.g.,
\code{-h} or {\code{--h}}, etc.

If both full and reduced \pkd\ outputs exist for the same timestep,
the reduced output will be used in place of the full output.

\section{EXAMPLE}

The following example generates a movie from every 10$^{\mathrm{th}}$
output frame, using POV-Ray to draw in 24-bit color, setting the frame
rate to 1 frame per second, and retaining the image files in the run
directory:

\begin{verbatim}
mkmov.py --keep -c -n 10 -r 1
\end{verbatim}

\end{document}
