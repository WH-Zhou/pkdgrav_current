\documentclass[12pt,dvips]{article}
\usepackage{graphicx}
\usepackage{amsmath}
\usepackage{amssymb}

\setlength{\topmargin}{0.0in}
\setlength{\headheight}{0.0in}
\setlength{\headsep}{0.0in}
\setlength{\oddsidemargin}{0.0in}
\setlength{\textheight}{9.0in}
\setlength{\textwidth}{6.5in}

\newcommand{\cf}{cf.\ }
\newcommand{\eg}{e.g., }
\newcommand{\etal}{et al.}
\newcommand{\ie}{i.e., }
\newcommand{\viz}{viz.\ }
\newcommand{\vs}{vs.\ }

\newcommand{\sect}[1]{Section \ref{s:#1}}
\newcommand{\fig}[1]{Fig.\ \ref{f:#1}}
\newcommand{\eqn}[1]{Eq.\ (\ref{e:#1})}

\newcommand{\code}[1]{\texttt{#1}}

\newcommand{\bvec}[1]{\mathbf{#1}} % vector
\newcommand{\bhat}[1]{\bvec{\hat{#1}}} % unit vector
\newcommand{\bmat}[1]{\bvec{#1}} % matrix
\newcommand{\bsym}[1]{\mbox{\boldmath $#1$}} % symbol (e.g., lowercase Greek)
\newcommand{\vdot}{\bsym{\cdot}} % dot product
\newcommand{\cross}{\bsym{\times}} % cross product
\newcommand{\trans}[1]{#1^\mathrm{T}} % transpose
\newcommand{\inver}[1]{#1^{-1}} % inverse

\newcommand{\hide}[1]{} % to omit text sections

\newcommand{\pkd}{\texttt{pkdgrav}}

\usepackage[hidelinks]{hyperref}

\begin{document}

\begin{flushleft}

\huge{SSDEM Documentation}\\
\bigskip\bigskip
\Large{Last Update: 7/7/20}\\
\bigskip\bigskip
\large{Started By: Derek C. Richardson}\\
\bigskip
\large{Contact Info:}\\
Department of Astronomy\\
University of Maryland\\
College Park MD 20742\\
Tel: 301-405-8786\\
E-mail: \texttt{dcr@astro.umd.edu}

\end{flushleft}

\tableofcontents

\section{OVERVIEW}

This is brief and incomplete documentation of the soft-sphere discrete
element method (SSDEM) implementation in \pkd.

\subsection{Other Documentation}

For documentation of walls in \pkd, see \code{walls.pdf}.

For documentation of \code{ssio.py}, a utility for reading and writing
binary \code{ss} files in \code{python}, see \code{ssio\_py.pdf}.

There is an SSDEM tutorial at \code{https://goo.gl/NIYJ5z} (or email
\code{dcr@astro.umd.edu}).

\section{USING A FIXED SEARCH BALL}

This section describes how to properly use the \code{DEM\_FIXED\_BALL}
compilation option, which can significantly reduce the time required
to run a simulation (depending on the circumstances). We first give an
introduction to the function and behavior of a search ball, then give
a description of the default dynamically generated search ball, and,
finally, describe the different fixed search ball options.

\textbf{If you already know which search ball option you want to use,
  simply read the first three lines of the sub-section of that option
  to understand what changes are required for proper use.}

\textbf{WARNING: Problems can occur with certain scenarios using a
  fixed search ball, such as a rubble pile asteroid flying by a
  planet.  In this case, the search ball will be huge (the size of the
  planet), so all the smaller particles will include each other in
  their search balls.  This can cause a parallel run to crash without
  warning due to a memory overflow.  In those cases, if the large
  particle is not intended to collide with the smaller ones, consider
  setting its radius to something much smaller.}

\subsection{Introduction}

In \code{pkdgrav}, each particle maintains a list of its nearest
neighbors in order to check for potential colliders. Rather than
performing a collision check for each other particle in a simulation,
which would take $N^2$ time, a particle only performs a collision
check on its nearest neighbors. The number of nearest neighbors that a
particle has to search for in order to effectively determine whether
it is in overlap with another particle largely depends on its size
with respect to other particles. For equal-sized particles, the
theoretical maximum number of neighbors that a single particle may be
in contact with is 12. (Note: this is only true if the particles are
``just touching'' and not overlapping significantly.) In general, for
a size distribution of particles, the maximum number of overlaps that
a single particle may experience is given by,
\begin{equation}
  \mathrm{MAX\_NUM\_OVERLAPS\_PER\_PARTICLE} = 8 + 4 \left(
  \frac{R_{\mathrm{max}}}{R_{\mathrm{min}}} \right)^2
  , \label{e:maxovlp}
\end{equation}
where $R_{\mathrm{max}}$ and $R_{\mathrm{min}}$ are the maximum and
minimum particle radii in the simulation, respectively. The value of
\code{MAX\_NUM\_OVERLAPS\_PER\_PARTICLE} is defined in the header file
\code{dem.h}.

Of vital importance to detecting neighbors, and therefore particle
contacts, is the use of a search ball by each particle. The search
ball of a particle defines the limited region around it where it will
look for nearest neighbors. In the default implementations of
\code{pkdgrav} and \code{SSDEM}, the search ball is dynamically
constructed each time step. Years of practical use of the codes have
shown that this might not be the most efficient way to find nearest
neighbors. Therefore, methods that generate a fixed search ball for
each particle make the code run faster as the search ball does not
need to be re-constructed each time-step. Three different fixed search
ball strategies were implemented in order to speed up processing time
and ensure proper neighbor finding for certain \code{pkdgrav}
scenarios. We outline these methods below, and best practices for
using them. First, we provide more context by reviewing the default
dynamic search ball method.

\subsection{Dynamic Search Ball}

For the default compilation of SSDEM, a particle's search ball is
dynamically set. The generation of the search ball is linked with the
partitioning of space into sub-domains and the building of the k-d
tree. A particle's search ball is thusly defined by the local number
density of particles.

For this dynamically generated search ball, the number of nearest
neighbors that each particle needs to keep track of is set by the
\code{nSmooth} parameter defined in the run-time parameter file
(\code{ss.par}). The value of \code{nSmooth} should be set, similar to
\eqn{maxovlp}, as,
\begin{equation}
  \mathrm{nSmooth} = 8 + 4 \left(
  \frac{R_{\mathrm{max}}}{R_{\mathrm{min}}} \right)^2
  . \label{e:nsmooth}
\end{equation}
This ensures that a small particle is able to detect the presence of a
larger particle in its vicinity. However, for a dynamic search ball,
\code{MAX\_NUM\_OVERLAPS\_PER\_PARTICLE} can be set to the theoretical
limit of 12 if the particles in the input file are size-sorted in
\emph{increasing} order. Size-sorting in increasing order allows the
smallest particle to first resolve any overlap conditions and apply
the necessary forces on itself and its larger collisional partners (by
applying Newtown's second law of motion). The collision algorithm then
progresses to the next largest particle. This next particle only looks
for collisions with particles larger than itself (since any collision
with smaller particles have already been calculated); hence, it too
only needs to look for 12 or fewer contacts. The algorithm progresses
through the list of particles. The largest particle need not look for
contacts as the forces from smaller particles have already been
calculated.

The advantage of using this search ball option is that it is general,
and its low memory use (only 12 colliders ever need to be stored into
memory). The disadvantage to this method is that the search ball needs
to be regenerated every time step, and, for a very large size
difference in particles, each particle will have to loop through
\code{nSmooth} nearest neighbors to check for overlaps. These two
factors require a lot of processing time, and poorly scale with total
number of particles.

\subsection{Fixed Search Ball}

By compiling \code{pkdgrav} with the \code{DEM\_FIXED\_BALL}
compilation option turned on, each particle will have a fixed search
ball to look for nearest neighbors. The most obvious advantage of a
fixed search ball is that processors spend less time re-computing
search balls each time step. The cost of using a fixed search ball is
larger memory requirements for certain cases. There are three options
for defining the size of the fixed search ball, controlled by the
\code{iFixedBallOption} defined in \code{ss.par}. These options are
described below.

\subsubsection{Maximum Diameter}

To use this option:
\begin{enumerate}
  \item Set \code{iFixedBallOption} to 0 in ss.par.
  \item Size-sort particles in increasing order.
  \item Set \code{MAX\_NUM\_OVERLAPS\_PER\_PARTICLE} = 12.
\end{enumerate}
This option sets the radius of each search ball to the diameter of the
largest particle. This option is optimal for equal-size particles, or
a size distribution of particles with a relatively small size ratio
($\frac{R_{\mathrm{max}}}{R_{\mathrm{min}}} \lesssim 3$). The
advantage of this option is the speed-up in having a fixed search
ball, while maintaining a relatively small memory requirement.

\subsubsection{Particle Radius $+$ Mean Radius}

To use this option:
\begin{enumerate}
  \item Set \code{iFixedBallOption} to 1 in ss.par.
  \item Size-sort particles in decreasing order.
  \item Use \eqn{maxovlp} to set
    \code{MAX\_NUM\_OVERLAPS\_PER\_PARTICLE}.
\end{enumerate}
With this option, the radius of each particle's search ball is set to
its own radius plus the mean radius of all particles. This allows each
particle to have a unique search ball that depends on its own
size. This option is especially useful for scenarios where a single
very large particle interacts with a large number of small particles
(of roughly equal size). This allows the large particle to efficiently
detect all the small particles as nearest neighbors, while also
limiting the size of the small particles search ball. This reduces the
number of nearest neighbors that each particle needs to check for
overlaps. However, the disadvantage is the large memory
requirement. Each particle needs to carry around an overlap list that
can contain \code{MAX\_NUM\_OVERLAPS\_PER\_PARTICLE} number of
particle IDs.

\subsubsection{Particle Radius $+$ Next-largest Particle Radius}

To use this option:
\begin{enumerate}
  \item Set \code{iFixedBallOption} to 2 in ss.par.
  \item Size-sort particles in decreasing order.
  \item Use \eqn{maxovlp} to set
    \code{MAX\_NUM\_OVERLAPS\_PER\_PARTICLE}.
\end{enumerate}
With this option, the radius of each particle's search ball is set to
its own radius plus the radius of the next-largest particle. This
option is useful when there is a large and continuous size
distribution of particles, or in the case of a single large intruder
interacting with a continuous size distribution of particles. This
optimizes the number of nearest neighbors a particle needs to check
for overlaps. The disadvantage is the large memory requirements.

\subsection{Sorting Particles by Radius}

Several utilities order particles in \code{ss} files by radius.  The
\code{ssgen} utility generates initial conditions that are
automatically saved in increasing radius order (or in decreasing order
with the \code{-v} option).  The \code{rpx} utility can be used to
sort an existing \code{ss} file by invoking the \code{-s} option (use
\code{rpx -h} to get a list of options).  Otherwise, particle data can
be altered manually by converting to \code{bt} files (see tutorial) or
by using the \code{ssio.py} script (see documentation).

\subsection{Search Ball Fudge Factor}

As a precaution, the routine that sets the search ball
(\code{dem.c:pkdDEMSetBall()}) includes a fudge factor
``\code{epsilon}'' (currently set to \code{0.1}) that is used when
calculating the search ball radius. Mathematically, this is how it is
applied for the three cases:
\begin{eqnarray*}
  0:\ R_{\mathrm{ball}} & = & R_{\mathrm{max}} + (1 + \epsilon) R_{\mathrm{max}} \\
  1:\ R_{\mathrm{ball}} & = & R_i + (1 + \epsilon) \bar{R} \\
  2:\ R_{\mathrm{ball}} & = & R_i + (1 + \epsilon) R_{i+1}
\end{eqnarray*}
(In the last case, it is assumed the file is in decreasing size order;
$R_{N+1}$ is just set to $R_N$.)

The utility \code{ssn}, which generates neighbor lists from \code{ss}
files, includes the three fixed-search-ball options above and has a
configurable fudge factor.  The utility can also be used to flag
particle overlaps that are detected within the search radius.

\section{RESTARTS WITH SSDEM} \label{s:restarts}

In SSDEM, particles keep track of overlaps (contacts) with other
particles (and walls, if applicable), but this information is not
stored in \code{ss} files or checkpoints.  Instead, special
\code{.dem} files are generated every full \code{ss} output, and these
can be used for clean restarts.  (In fact, checkpointing is disabled
altogether when running \pkd\ with SSDEM.)

To restart a simulation from a full \code{ss} output with a
corresponding \code{.dem} file, use the following procedure:
\begin{enumerate}
\item Set \code{achInFile} in \code{ss.par} (or equivalent
  \pkd\ parameter file) to the \code{ss} output that will be used, \eg
  \code{ss.01000}.
\item Change \code{iStartStep} to the timestep of the output (1000 in
  this example).
\item Set \code{bReadDEMData} to 1.
\item Rerun with the \code{+overwrite} option.
\end{enumerate}
Note that for this procedure the following preprocessor macros cannot
be changed between restarts:
\begin{itemize}
\item \code{USE\_WALLS}.
\item \code{USE\_DEM\_ROTATION\_DASHPOT}.
\item \code{MAX\_NUM\_OVERLAPS\_PER\_PARTICLE} (in \code{dem.h}).
\item \code{MAX\_NUM\_OVERLAPS\_PER\_PARTICLE\_FOR\_WALLS} (in
  \code{dem.h}).
\end{itemize}
Generally speaking, any changes to run parameters should also be
considered carefully to avoid any inconsistencies on restarts.  This
includes changing the number or order of walls, if applicable.

The frequency of outputting \code{.dem} files is set by the
\code{iOutInterval} parameter.  The \code{.dem} files are big, so you
can limit how many are kept using the \code{nDEMOutputs} parameter
(the default is 1, corresponding to keeping only the latest
\code{.dem} file).

\subsection{Format of \code{.dem} Files}

A \code{.dem} file is a binary file consisting of a header with the
following ordered information:
\begin{itemize}
\item Time (double).
\item Number of particles (integer).
\item \code{MAX\_NUM\_OVERLAPS\_PER\_PARTICLE} (integer).
\item \code{MAX\_NUM\_OVERLAPS\_PER\_PARTICLE\_FOR\_WALLS} (integer;
  $-1$ if \code{WALLS} is not defined).
\item Rotation dashpot flag (integer; 1 if
  \code{DEM\_ROTATION\_DASHPOT} is defined, 0 otherwise).
\end{itemize}
This is followed by an ordered data record, one for each
particle, of format:
\begin{itemize}
\item Predicted velocity (vector).
\item Predicted spin (vector).
\item \code{MAX\_NUM\_OVERLAPS\_PER\_PARTICLE} repetitions of:
  \begin{itemize}
  \item \code{iOrder} of overlapping particle (integer; $-1$ if none).
  \item Tangential displacement (shear) from initial contact (vector).
  \item Previous contact normal (vector).
  \item If \code{DEM\_ROTATION\_DASHPOT} defined:
    \begin{itemize}
    \item Rolling angular displacement (vector).
    \item Twisting scalar displacement (double).
    \end{itemize}
  \item Overlap counter (4-byte integer).
  \end{itemize}
\item If \code{WALLS} defined,
  \code{MAX\_NUM\_OVERLAPS\_PER\_PARTICLE\_FOR\_WALLS} repetitions of:
  \begin{itemize}
  \item ID of overlapping wall (integer; $-1$ if none).
  \item Tangential displacement (shear) from initial contact (vector).
  \item Previous contact normal (vector).
  \item If \code{DEM\_ROTATION\_DASHPOT} defined:
    \begin{itemize}
    \item Rolling angular displacement (vector).
    \item Twisting scalar displacement (double).
    \end{itemize}
  \item Overlap counter (4-byte integer).    
  \end{itemize}
\end{itemize}
Note this information is also given in the comments of
\code{src/pkdgrav/dem.h} related to \code{DEMHEAD\_SIZE} and
\code{DEMDATA\_SIZE}.

\section{DIAGNOSTIC OUTPUTS}

Currently there are 2 sets of diagnostics that can be generated for
SSDEM runs.

\subsection{SSDEM Statistics}

At a frequency set by the \code{iDEMStatsInterval} parameter, basic
SSDEM statistics can be output in \code{.demstats} files.  These are
text files that contain a header describing their contents; in brief,
the data stored include histograms of particle overlaps, overlap
orientations (``$\cos{\alpha}$''), and tangential spring $\bvec{S}$
magnitudes.  If \code{sm} and \code{ffmpeg} available, the file
\code{demstats.sm} can be used to visualize the data; copy the file
(from the \pkd\ \code{etc} directory) to the working directory and run
the \code{demstatsanim} script.  This will generate snapshots of the
histograms and stitch them together into 3 corresponding movies
(\code{demstats\_ohist.mp4}, \code{demstats\_ohist.mp4}, and
\code{demstats\_ohist.mp4}, respectively).

Important: accumulated histogram data is lost on a restart.  Also,
only particle-particle data is stored, not particle-wall.

\subsection{SSDEM Forces}

If \code{USE\_DEM\_DIAG} is defined in \code{Makefile.in}, at every
full \code{ss} output a \code{.demdiag} file will be generated.  This
is a text file with the following format:
\begin{verbatim}
N
OP [ OW ]
i1 Fnx Fny Fnz Ftx Fty Ftz i2 ... [ w1 Fnx Fny Fnz Ftx Fty Ftz w2 ... ]
i1 ...
...
\end{verbatim}
The first line gives the number of particles (\code{N}).
\code{MAX\_NUM\_OVERLAPS\_PER\_PARTICLE} is encoded as \code{OP} on
the second line, along with \code{OW} for
\code{MAX\_NUM\_OVERLAPS\_PER\_PARTICLE\_FOR\_WALLS}, if applicable,
in case these values change.  Each subsequent line (one per particle)
gives a list of overlaps for each particle: \code{i1}, \code{i2},
etc., are the \code{iOrder} numbers of the overlapped particles for
that particle while \code{Fn} and \code{Ft} are the \code{x},
\code{y}, \code{z} components of the normal and tangential forces due
to that overlap, respectively; \code{w1}, \code{w2}, etc., are the
wall id numbers (starting from 0) for wall overlaps (if
\code{USE\_WALLS} is defined in \code{Makefile.in}), with
corresponding force components as for particles.  There are \code{OP}
particle overlap entries per particle and \code{OW} wall overlap
entries per particle (see \code{dem.h}).

Important: particles only store overlap information for particles of
higher \code{iOrder} number.  For example, if particles 3 and 5 are
overlapping, only particle 3 stores the information.  As a result,
only particle 3 in the \code{.demdiag} file will contain information
on the overlap with particle 5.  It is up to the user to reconstruct
the overlap data for particle 5 due to particle 3 (which will be just
the negative of the force components for particle 3 due to particle 5,
by Newton's 3$^{\mathrm{rd}}$ law).

\section{TRACKING SPHERE ORIENTATIONS}

If \code{pkdgrav} is compiled with the \code{DEM\_TRACK\_ORIENT}
option, the orientations of spherical particles will be tracked
explicitly and stored in files for analysis and plotting.  If no
initial orientations are specified (see \code{bReadOrient} parameter
below), all particles are assumed to start with orientations aligned
with the space axes.  Currently particles in aggregates or stuck to
walls will not have orientations updated, and runs using
\code{SLIDING\_PATCH} will not do the update properly (\ie the
orientations will not account correctly for the rotating frame).
Orientations are output to binary ``\code{.ori}'' files at both the
full output (double precision) and reduced output (single precision)
frequency.  The utilities \code{ori2ort} and \code{ort2ori} are
provided to convert between binary (\code{.ori}) and text
(\code{.ort}) representations.  The \code{.ort} files consist of 1
particle per line in the following format:
\begin{displaymath}
  \centering
  \begin{array}{llllll}
    p_{1,x} & p_{1,y} & p_{1,z} & p_{2,x} & p_{2,y} & p_{2,z}
  \end{array}
\end{displaymath}
where $\bhat{p}_i = (p_{i,x},p_{i,y},p_{i,z})$ is principal axis $i$
of the orientation matrix, a unit vector.  Only $\bhat{p}_1$ and
$\bhat{p}_2$ are stored because $\bhat{p}_3 = \bhat{p}_1 \cross
\bhat{p}_2$ by virtue of the orientation matrix being orthonormal.  It
might be possible to omit one more element, but the added overhead of
reconstruction does not make this worthwhile.

Because the particles have spherical symmetry, the Euler
equations\footnote{\code{http://www.astro.umd.edu/$\sim$dcr/reprints/richardson\_icarus115,320.pdf}}
are particularly simple to solve, with spin vectors updated during
leapfrog kicks (as they were before this new feature) and orientations
updated during drifts (with a call to \code{dem.c:demUpdateOrient()}).
The torques that are used to update the spins are computed as normal
in the DEM routines along with other forces.

\subsection{Initial Conditions and Restarts}

If the \code{bReadOrient} parameter has a non-zero value,
\code{pkdgrav} will read the \code{.ori} file associated with
\code{achInFile} on start-up (\eg if the input file is
\code{ss.12345}, \code{pkdgrav} will look for \code{ss.12345.ori}).
This file must be in standard format (not reduced).  This can be used
to give the particles a particular orientation at the start (see the
format above), or for restarts (\sect{restarts}).

\subsection{Drawing Sphere Orientations}

If invoked with the \code{-o} option, \code{ssdraw} will read the
orientation (\code{.ori}) files associated with each \code{ss} file
and apply the appropriately rotated \code{StyleOrient} texture (in
\code{povray.inc}).  This only works for \code{POV-Ray} output.  A few
patterns are provided in the template \code{povray.inc} file (located
in the \code{pkdgrav} \code{etc} folder), though currently only 1
pattern can be chosen at a time, which is applied to all particles.
Transparency specified in the pattern pigment can be used to bring out
the underlying particle color.  To make a movie with particle
orientations, use \code{mkmov.py -o}.

\subsection{Future Updates}

\begin{enumerate}
\item Add support for \code{AGGS}, sticky walls, and
  \code{SLIDING\_PATCH}.
  \begin{itemize}
  \item For \code{AGGS}, the particle orientation could be transformed
    to the aggregate body frame and frozen in until the inertia tensor
    changes (due to adding or removing of particles). Transform back
    to the space frame for output.
  \item For sticky walls, the particle orientation for reactive walls
    could be treated the same as for \code{AGGS}, while for
    non-reactive walls the needed transformation can be computed
    within \code{ssdraw}.
  \item For \code{SLIDING\_PATCH}, it is necessary to account for the
    rotating frame. Code to do that for the particle spins is already
    in \code{pkdgrav}; that code could be leveraged to also update the
    particle orientations, in principle.
  \end{itemize}
\end{enumerate}
  
\section{OTHER SSDEM NOTES}

\subsection{Softening}

In its original incarnation, \code{pkdgrav} used softening to prevent
interparticle forces from getting arbitrarily large during a close
encounter.  With \code{COLLISIONS}, the softening radius is set equal
to the particle radius, largely because the two concepts are
considered mutually exclusive so the same particle data structure
element could be used for both (see the \code{RADIUS} macro in
\code{collision.h}).  In \code{HSDEM}, this works well because
particles are not supposed to interpenetrate.  But in \code{SSDEM},
particles are allowed to interpenetrate, and this determines the
restoring and friction forces.  This means that in simulations where
particles experience self-gravity (say a self-gravitating rubble
pile), gravity forces will be softened when particles overlap.  In
practice this should be a small effect, since overlaps are meant to be
kept below 1\% of the smallest particle radius, but if this condition
is relaxed, or extreme fidelity is needed for some reason, the
softening would need to be taken into account (and may necessitate a
new scheme, such as making the particle radius twice the softening
radius, etc.).

\end{document}
