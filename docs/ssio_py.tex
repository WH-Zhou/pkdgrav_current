\documentclass[12pt,dvips]{article}
\usepackage{graphicx}
\usepackage{amsmath}
\usepackage{amssymb}

\setlength{\topmargin}{0.0in}
\setlength{\headheight}{0.0in}
\setlength{\headsep}{0.0in}
\setlength{\oddsidemargin}{0.0in}
\setlength{\textheight}{9.0in}
\setlength{\textwidth}{6.5in}

\newcommand{\cf}{cf.\ }
\newcommand{\eg}{e.g., }
\newcommand{\etal}{et al.}
\newcommand{\ie}{i.e., }
\newcommand{\viz}{viz.\ }
\newcommand{\vs}{vs.\ }

\newcommand{\sect}[1]{Section \ref{s:#1}}
\newcommand{\fig}[1]{Fig.\ \ref{f:#1}}
\newcommand{\eqn}[1]{Eq.\ (\ref{e:#1})}

\newcommand{\code}[1]{\texttt{#1}}

\newcommand{\bvec}[1]{\mathbf{#1}} % vector
\newcommand{\bhat}[1]{\bvec{\hat{#1}}} % unit vector
\newcommand{\bmat}[1]{\bvec{#1}} % matrix
\newcommand{\bsym}[1]{\mbox{\boldmath $#1$}} % symbol (e.g., lowercase Greek)
\newcommand{\vdot}{\bsym{\cdot}} % dot product
\newcommand{\cross}{\bsym{\times}} % cross product
\newcommand{\trans}[1]{#1^\mathrm{T}} % transpose
\newcommand{\inver}[1]{#1^{-1}} % inverse

\newcommand{\hide}[1]{} % to omit text sections

\newcommand{\pkd}{\texttt{pkdgrav}}


\begin{document}

\begin{flushleft}

\huge{\pkd\ \code{ssio.py} Documentation}\\
\bigskip\bigskip
\Large{Last Update 7/17/19}\\
\bigskip\bigskip
\large{Created by Ron Ballouz 11/25/15}\\

\end{flushleft}

\section{OVERVIEW}

The \code{ssio.py} library contains the Python
functions \code{read\_SS()} and \code{write\_SS()} that allow the
reading and writing of \code{ss} files in a Python script.  In order
to work correctly, \code{ssio.py} requires Python 3 and the NumPy
library.

\section{Setup}
In order to use the functions in the \code{ssio.py} library, the file
(located in the \pkd\ codebase under \code{bin/scripts}) must be
copied to a local directory, or the path to the file can be placed in
the \code{PYTHONPATH} environment variable.  For example, in bash:
\begin{verbatim}
export PYTHONPATH=/path_to_pkdgrav/bin/scripts
\end{verbatim}
where \code{path\_to\_pkdgrav} is the location of the \pkd\ codebase.
Afterwards, one can import the \code{ssio.py} library from any Python
script via:
\begin{verbatim}
import ssio
\end{verbatim}

\section{\code{ssio.read\_SS()}}
The \code{read\_SS()} function allows a reduced or full \pkd\ output
file to be read in by a Python script.  \code{read\_SS()} returns
either a $14 \times N$ NumPy array (where $N$ is the number of
particles) of the \pkd\ data file, or, if the optional
argument \code{unpack=True}, a list of individual NumPy arrays.  The
function can optionally return the header information as well.  As an
argument, \code{read\_SS()} requires the input file name, and
optionally takes a string to specify whether the header information of
the file should also be returned.  The returned header is a list of
three values: the timestamp, the number of particles, and the ``magic
number'' ($-1$ for full output and $-10101$ for reduced).  For example:
\begin{verbatim}
import ssio

#to get the header information and the data
header, data = ssio.read_SS('my_file.ss', 'yes')
#note this second argument should be any string that is not 'no'

#to just get the data
data = ssio.read_SS('my_file.ss')
\end{verbatim}
If an error is encountered during reading, the \code{ssio.ReadError}
exception is raised, with an explanatory message available.  To handle
the exception cleanly and print the error message, you could do the
following:
\begin{verbatim}
import ssio

try:
    data = ssio.read_SS('my_file.ss')
except ssio.ReadError as msg:
    print('Read error:', msg)
    exit(1)
\end{verbatim}
The data order is as follows: order number (integers from 0 to $N -
1$); original order number (or original index); mass (doubles); radius
(doubles); position (as a vector of doubles); velocity (doubles
vector); spin (doubles vector); and color (integers).  Here is sample
usage with \code{unpack=True}:
\begin{verbatim}
import ssio
header, ord, oord, mass, rad, pos, vel, spin, c = \
    ssio.read_SS('my_file.ss', 'yes', unpack=True)
print(pos[:, 0])  # vector position of first particle
\end{verbatim}

\section{\code{write\_SS()}}
The \code{write\_SS()} function allows the user to write a $14 \times
N$ NumPy array to a reduced or full output \pkd\ file.  As
arguments, \code{write\_SS()} takes a NumPy array and the name of the
output file.  Optionally, \code{write\_SS()} can accept a time to
write to the output file's header.  \code{write\_SS()} will write to a
reduced output file if the filename given has a ``.r'' extension.  For
example, to read in a file (specified as a command-line argument),
change all the particle colors to red, and then save to a reduced
output file:
\begin{verbatim}
import sys
import ssio

file = sys.argv[1]
header, py_data = read_SS(file, 'yes')
py_data[13, :] = 2
ssio.write_SS(py_data, file + '_recolored.r', time = header[0])
\end{verbatim}
If an error is encountered during writing, the \code{ssio.WriteError}
exception is raised, with an explanatory message available,
analogously to exception handling for \code{read\_SS()}.

\end{document}
