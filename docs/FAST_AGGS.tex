\documentclass[12pt,dvips]{article}
\usepackage{graphicx}
\usepackage{amsmath}
\usepackage{amssymb}

\setlength{\topmargin}{0.0in}
\setlength{\headheight}{0.0in}
\setlength{\headsep}{0.0in}
\setlength{\oddsidemargin}{0.0in}
\setlength{\textheight}{9.0in}
\setlength{\textwidth}{6.5in}

\newcommand{\cf}{cf.\ }
\newcommand{\eg}{e.g., }
\newcommand{\etal}{et al.}
\newcommand{\ie}{i.e., }
\newcommand{\viz}{viz.\ }
\newcommand{\vs}{vs.\ }

\newcommand{\sect}[1]{Section \ref{s:#1}}
\newcommand{\fig}[1]{Fig.\ \ref{f:#1}}
\newcommand{\eqn}[1]{Eq.\ (\ref{e:#1})}

\newcommand{\code}[1]{\texttt{#1}}

\newcommand{\bvec}[1]{\mathbf{#1}} % vector
\newcommand{\bhat}[1]{\bvec{\hat{#1}}} % unit vector
\newcommand{\bmat}[1]{\bvec{#1}} % matrix
\newcommand{\bsym}[1]{\mbox{\boldmath $#1$}} % symbol (e.g., lowercase Greek)
\newcommand{\vdot}{\bsym{\cdot}} % dot product
\newcommand{\cross}{\bsym{\times}} % cross product
\newcommand{\trans}[1]{#1^\mathrm{T}} % transpose
\newcommand{\inver}[1]{#1^{-1}} % inverse

\newcommand{\hide}[1]{} % to omit text sections

\newcommand{\pkd}{\texttt{pkdgrav}}


\begin{document}

\begin{flushleft}

\huge{\pkd\ \code{FAST\_AGGS} Documentation}\\
\bigskip\bigskip
\Large{Last Update 6/3/20}\\
\bigskip\bigskip
\large{Created by Joe DeMartini 9/16/19}\\

\end{flushleft}

\section{OVERVIEW}

\code{FAST\_AGGS} is a \code{pkdgrav} compile option that drastically speeds up aggregate 
particle calculations performed in the code, under the condition that the number of free 
bodies is comparable to the number of aggregates, by using a binary search algorithm, 
followed by a cache line approach which exploits particle ordering on the processor, to find 
an aggregate's constituent particles, rather than doing a brute-force search, as in the old 
code. The specific use criterion is: the \code{FAST\_AGGS} compile option should not be used 
when there are very few aggregates made up of very many particles in the problem, but 
\textbf{only when there are very many aggregates each containing very few particles}. Only in 
the latter case will the efficiency increase when using \code{FAST\_AGGS}.\\

\medskip

\underline{Notes/Requirements}:
\begin{enumerate}
\item \code{FAST\_AGGS} is not compatible with the \code{AGGS\_IN\_PATCH} compile option.
\item \code{FAST\_AGGS} requires that particles and aggregates be in the same order on the
processor: if aggregate 0 is a dumbbell, it must contain particles with iOrder number 0 and 1,
aggregate 1 must contain particles with iOrder 2 and 3, etc. This is a key stipulation of the
updated searching algorithm in order to gain maximum increased efficiency, and is not always
upheld when aggregates are added using \code{rpx}.
\item \code{FAST\_AGGS} requires that, in simulations with both aggregates and single spheres,
the particles in aggregates be listed first on the processor, followed by the individual
spherical particles. When ordering particles in these types of simulations, be careful to
abide by the above ordering rule in bullet 2, and the necessary size ordering for
\code{DEM\_FIXED\_BALL} (See SSDEM.pdf page 4). 
\item Different options of \code{DEM\_FIXED\_BALL} require different kinds of particle size
ordering. Since aggregates must be in iOrder number order for this algorithm, aggregates
constructed of different-sized particles are not compatible with both \code{FAST\_AGGS} and
\code{DEM\_FIXED\_BALL}.
\end{enumerate}
\medskip
For a specific, well-commented example of how the new code is written, see 
\code{pkdAggsGetAccel} and its associated functions in \code{SOURCE/src/pkdgrav/aggs.c}.

\section{Setup}
In order to make use of \code{FAST\_AGGS}, one must uncomment both of
the following options in Makefile.in:
\begin{verbatim}
USE_AGGREGATES=true
USE_FAST_AGGS=true
\end{verbatim}
Then one can recompile the code and the new routines will take effect.

Again, we note that \code{FAST\_AGGS} is not compatible with the
\code{AGGS\_IN\_PATCH} compile option. Otherwise, all restrictions
that previously applied to Aggregates will apply here as well.

\section{Backwards Compatibility}
\code{FAST\_AGGS} is entirely backwards-compatible with the older aggregate
routines in \pkd. To use the older aggregate routines, simply leave the
\code{USE\_FAST\_AGGS=true} option commented in Makefile.in, while uncommenting
\code{USE\_AGGREGATES=true}. This will revert to the older versions of the
functions used for aggregate calculations.

\section{Affected Functions}
\begin{verbatim}
pkdAggsCountPart
pkdAggsSetSpacePos
pkdAggsSetSpaceVel
pkdAggsSetSpaceSpins
pkdAggsGetAccel
pkdAggsGetTorque
pkdAggsSetBodyPos
pkdAggsGetCOM
pkdAggsGetAxesAndSpin

pkdAggsSetMassDEM (formerly pkdDEMAggsSetMass)
pstAggsSetMassDEM

msrAggsGravity
msrAggsFind 
\end{verbatim} 

The first set are the functions to which a binary search was added for improved efficiency. 
The first function in the second set had a binary search was added, and both functions were 
renamed, updated, and moved to files/locations where they fit better (\code{pkdAggsSetMassDEM}
to \code{aggs.c} and \code{pstAggsSetMassDEM} changed and updated in \code{pst.c} under the
\code{AGGS} banner, rather than \code{DEM}). The last set are functions in \code{master.c} to
which a call to \code{msrReorder} was added, so that the particles are in aggregate number
order on the processor - this is an important step for the new algorithm that searches
for particles in an aggregate.


\end{document}
